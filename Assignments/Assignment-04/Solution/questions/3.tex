در ابتدا کد داده شده را به $3AC$ تبدیل می‌کنیم:
\setLTR
\begin{lstlisting}
x:=10
y:=20
if x>y goto 5 
goto 7
z:=x+y
goto 8
w:=x-y
result:=y-x
\end{lstlisting}
\setRTL
سپس مقدار عددی متغیرها را جایگزین می‌کنیم:
\setLTR
\begin{lstlisting}
x:=10
y:=20
if 10>20 goto 5 
goto 7
z:=10+20
goto 8
w:=10-20
result:=20-10
\end{lstlisting}
\setRTL
حال عملیات‌های ریاضی را ساده‌سازی می‌کنیم:
\setLTR
\begin{lstlisting}
x:=10
y:=20
if 10>20 goto 5 
goto 7
z:=30
goto 8
w:=-10
result:=10
\end{lstlisting}
\setRTL
چون شرط همواره غلط است، آن را حذف می‌کنیم:
\setLTR
\begin{lstlisting}
x:=10
y:=20

goto 7
z:=30
goto 8
w:=-10
result:=10
\end{lstlisting}
‌\setRTL

خط‌های 5 و 6 غیر قابل دسترس هستند، پس آن‌ها را حذف می‌کنیم: 
\setLTR
\begin{lstlisting}
x:=10
y:=20

goto 7


w:=-10
result:=10
\end{lstlisting}
\pagebreak
\setRTL
نیازی به دستور $goto$ نداریم پس حذفش می‌کنیم:
\setLTR
\begin{lstlisting}
x:=10
y:=20




w:=-10
result:=10
\end{lstlisting}
\setRTL
وجود متغیر $w$ اضافی است، پس حذفش می‌کنیم:
\setLTR
\begin{lstlisting}
x:=10
y:=20





result:=10
\end{lstlisting}
\setRTL
وجود متغیرهای $x,y$ اضافی است، پس حذفشان می‌کنیم و می‌‌بینیم که کد به ساده‌ترین شکل ممکن نوشته شد:
\setLTR
\begin{lstlisting}







result:=10
\end{lstlisting}
\setRTL

البته صورت سوال صرفا خواسته بود که کد مرده شناسایی و حذف شود، در این صورت می‌شد $x,y$ را حذف نکرد و $w$ را برحسب آن دو نوشت. اما این حالت بهینه‌ترین نیست.