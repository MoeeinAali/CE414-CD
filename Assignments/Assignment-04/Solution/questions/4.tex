در ابتدا کد را سه آدرسه مینویسیم:
\setLTR
\begin{lstlisting}
x:=2
y:=3
z:=x+y
x:=4
w:=z*2
\end{lstlisting}
\setRTL
حال مقدار عددی متغیرها را جایگزین کرده:
\setLTR
\begin{lstlisting}
x:=2
y:=3
z:=2+3
x:=4
w:=z*2
\end{lstlisting}
\setRTL
حال عملیات جمع را ساده کرده و ضرب را به شیفت تبدیل میکنیم:
\setLTR
\begin{lstlisting}
x:=2
y:=3
z:=5
x:=4
w:=z<<1
\end{lstlisting}
\setRTL
متغیرهای x و y اضافی هستند، آنها را حذف میکنیم:
\setLTR
\begin{lstlisting}


z:=5

w:=z<<1
\end{lstlisting}
\setRTL
حال مقدار z را جایگزین میکنیم:
\setLTR
\begin{lstlisting}




w:=5<<1
\end{lstlisting}
\setRTL
و در آخر عبارت شیفت را ساده میکنیم:
\setLTR
\begin{lstlisting}




w:=10
\end{lstlisting}
\setRTL